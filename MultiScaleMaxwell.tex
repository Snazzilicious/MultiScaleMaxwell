
\documentclass{article}

\usepackage[a4paper, total={6in, 8in}]{geometry}
\usepackage{amsmath}
\usepackage{amsfonts}
\usepackage{amsthm}
\usepackage{indentfirst}
\usepackage{hyperref}

\newcommand{\norm}[1]{||#1||}
\newcommand{\Div}[0]{\nabla\cdot}
\newcommand{\Curl}[0]{\nabla\times}

\theoremstyle{plain}
\newtheorem*{remark}{Remark}

\begin{document}

\section{Intro}\label{sec:intro}

We begin with Maxwell's equations in matter
\begin{equation}
	\Div\mathbf{D} = \rho
\end{equation}
\begin{equation}
	\Div\mathbf{B} = 0
\end{equation}
\begin{equation}
	\Curl\mathbf{E} = -\frac{d\mathbf{B}}{dt}
\end{equation}
\begin{equation}
	\Curl\mathbf{H} = \mathbf{J} + \frac{d\mathbf{D}}{dt}.
\end{equation}
We assume linearly polarizable media,
\begin{equation}
	\mathbf{D} = \epsilon\mathbf{E}
\end{equation}
\begin{equation}
	\mathbf{H} = \frac{1}{\mu}\mathbf{B},
\end{equation}
where $\epsilon=\epsilon(x)$ and $\mu=\mu(x)$. We also assume the solution to be time-harmonic (e.g. $\mathbf{E}(t,x)=\mathbf{E}(x)e^{-j\omega t}$) and so replace time derivative operators with $-j\omega$. We furthermore assume $\mathbf{E}(x) = \mathbf{e}(x)e^{j\omega S(x)}$ and $\mathbf{H}(x) = \mathbf{h}(x)e^{j\omega S(x)}$. The resulting equations are
%\begin{equation}\label{eq:max1}
%	\Div\left(\epsilon\mathbf{E}\right) = \rho
%\end{equation}
%\begin{equation}\label{eq:max2}
%	\Div\mathbf{B} = 0
%\end{equation}
%\begin{equation}\label{eq:max3}
%	\Curl\mathbf{E} = -j\omega\mathbf{B}
%\end{equation}
%\begin{equation}\label{eq:max4}
%	\Curl\left(\frac{1}{\mu}\mathbf{B}\right) = \mathbf{J} + j\omega\epsilon\mathbf{E}.
%\end{equation}
% NOTE: Using SI units instead of Gauss
\begin{equation}\label{eq:max1}
	\mathbf{e}\cdot\nabla S = \frac{1}{j\omega\epsilon}\left( \rho e^{-j\omega S(x)} - \Div\epsilon\mathbf{e} \right)
\end{equation}
\begin{equation}\label{eq:max2}
	\mathbf{h}\cdot\nabla S = -\frac{1}{j\omega\mu}\Div\mu\mathbf{h}
\end{equation}
\begin{equation}\label{eq:max3}
	\nabla S\times\mathbf{e} - \mu\mathbf{h} = -\frac{1}{j\omega}\Curl\mathbf{e}
\end{equation}
\begin{equation}\label{eq:max4}
	\nabla S\times\mathbf{h} + \epsilon\mathbf{e} = \frac{1}{j\omega}\left( \mathbf{J}e^{-j\omega S(x)} - \Curl\mathbf{h} \right).
\end{equation}
So far this closely follows Wolf and Born, except using SI units instead of Gauss, and retaining charges and currents.


\section{First Order System}

Leaving the system of equations as is, we have two systems
\begin{equation}
	\omega^0:
	\begin{cases}
		\mathbf{e}\cdot\nabla S = 0 \\
		\mathbf{h}\cdot\nabla S = 0 \\
		\nabla S\times\mathbf{e} - \mu\mathbf{h} = 0 \\
		\nabla S\times\mathbf{h} + \epsilon\mathbf{e} = 0
	\end{cases}
\end{equation}
\begin{equation}
	\omega^1:
	\begin{cases}
		\Div\epsilon\mathbf{e} = \rho' \\
		\Div\mu\mathbf{h} = 0 \\
		\Curl\mathbf{e} = 0 \\
		\Curl\mathbf{h} = \mathbf{J}',
	\end{cases}
\end{equation}
where we have defined $\rho' = \rho e^{-j\omega S(x)}$ and $\mathbf{J}' = \mathbf{J}e^{-j\omega S(x)}$.

To solve the first system, we first solve the third equation for $\mathbf{h}$, then plug that into the fourth equation. This gives
\begin{equation}
	\frac{1}{\mu}\left[ \left(\mathbf{e}\cdot\nabla S\right)\nabla S - \norm{\nabla S}^2\mathbf{e}\right] + \epsilon\mathbf{e} = 0.
\end{equation}
Invoking the first equation results in
\begin{equation}
	\left(\mu\epsilon - \norm{\nabla S}^2\right)\mathbf{e} = 0,
\end{equation}
or
\begin{equation}
	\norm{\nabla S}^2 = \mu\epsilon.
\end{equation}
This is the eikonal equation.

Inspecting the second system, we see that it is the plain electro/magneto-statics equations, that is to say that the wave part of the solution has been removed. Hopefully this is easier to solve than the EFIE/MFIE problem.

We start by considering the fourth equation's implication that $\Div\mathbf{J}'=0$. This means that
\begin{equation}
	\Div e^{-j\omega S}\mathbf{J} = e^{-j\omega S}\Div\mathbf{J} - j\omega e^{-j\omega S}\mathbf{J}\cdot\nabla S = 0,
\end{equation}
or
\begin{equation}
	e^{-j\omega S}\Div\mathbf{J} = j\omega e^{-j\omega S}\mathbf{J}\cdot\nabla S.
\end{equation}
Invoking $\rho = \frac{1}{j\omega}\Div\mathbf{J}$, we have $\rho' = \mathbf{J}'\cdot\nabla S$, so the first equation is
\begin{equation}
	\Div\epsilon\mathbf{e} = \mathbf{J}'\cdot\nabla S
\end{equation}
The third equation implies $\mathbf{e} = \nabla\phi$ for some scalar function $\phi$. The first equation then becomes
\begin{equation}
	\Div\epsilon\nabla\phi = \mathbf{J}'\cdot\nabla S.
\end{equation}
This would produce a solution for the electric field
\begin{equation}
	\mathbf{E}(x) = e^{j\omega S}\nabla\phi.
\end{equation}
The second equation implies $\mu\mathbf{h} = \Curl\mathbf{A}$ for some vector function $\mathbf{A}$. The fourth equation then becomes
\begin{equation}
	-\nabla^2\mathbf{A} -\nabla\log{\mu}\times\Curl\mathbf{A} = \mu\mathbf{J}',
\end{equation}
where an appropriate gauge has been chosen such that $\Div\mathbf{A}=0$. The solution for the magnetic field is
\begin{equation}
	\mathbf{H}(x) = \frac{e^{j\omega S}}{\mu}\Curl\mathbf{A}.
\end{equation}


Of particular interest are solutions when $\mathbf{J} = \delta(x-x_0)$. We will indicate these solutions by $\phi(x,x_0)$ and $\mathbf{A}(x,x_0)$. It is convenient parameterize the phase function by the center of the delta function, $S=S(x,x_0)$, and to require that $S(x_0,x_0)=0$ so that $\mathbf{J}'(x_0)=\mathbf{J}(x_0)$. With all this, the solution to the general problem is
\begin{equation}
	\mathbf{E}(x) = \int_{\mathbb{R}^3} e^{j\omega S(x,x_0)} \left( \mathbf{J}(x_0)\cdot\nabla S(x,x_0) \right)\nabla_x\phi(x,x_0) dx_0
\end{equation}
and
\begin{equation}
	\mathbf{H}(x) = \frac{1}{\mu}\int_{\mathbb{R}^3} e^{j\omega S(x,x_0)}\nabla_x\times\left( \mathbf{J}(x_0)\circ\mathbf{A}(x,x_0) \right) dx_0.
\end{equation}


% TODO Update this
\subsection{Homogeneous Media}

For reference, we consider the case of constant permeability and permittivity. In this case we can set the phase function to
\begin{equation}
	S(x,x_0) = \sqrt{\mu\epsilon}\norm{x-x_0}
\end{equation}
and the equations to solve are
\begin{equation}
	\nabla^2\phi = \sqrt{\frac{\mu}{\epsilon}}\frac{\mathbf{J}_0\cdot(x-x_0)}{\norm{x-x_0}}\delta(x-x_0)
\end{equation}
and
\begin{equation}
	\nabla^2\mathbf{A} = -\mu\mathbf{J}_0\delta(x-x_0).
\end{equation}
Both of these are Poisson's equation which has Green's function
\begin{equation}
	G(x,x_0) = \frac{-1}{4\pi\norm{x-x_0}},
\end{equation}
the gradient of which is
\begin{equation}
	\nabla G(x,x_0) = \frac{x-x_0}{4\pi\norm{x-x_0}^3}.
\end{equation}
The solutions are thus
\begin{equation}
	\mathbf{E}(x) = e^{j\omega\sqrt{\mu\epsilon}\norm{x-x_0}}\frac{\rho_0(x-x_0)}{4\pi\epsilon\norm{x-x_0}^3}
\end{equation}
and
\begin{equation}
	\mathbf{H}(x) = e^{j\omega\sqrt{\mu\epsilon}\norm{x-x_0}}\frac{\mathbf{J}_0\times(x-x_0)}{4\pi\norm{x-x_0}^3}.
\end{equation}
From these we see that the general solutions are
\begin{equation}
	\mathbf{E}(x) = \int_{\mathbb{R}^3} e^{j\omega\sqrt{\mu\epsilon}\norm{x-x_0}}\frac{\rho(x_0)(x-x_0)}{4\pi\epsilon\norm{x-x_0}^3} dx_0
\end{equation}
and
\begin{equation}
	\mathbf{H}(x) = \int_{\mathbb{R}^3} e^{j\omega\sqrt{\mu\epsilon}\norm{x-x_0}}\frac{\mathbf{J}(x_0)\times(x-x_0)}{4\pi\norm{x-x_0}^3} dx_0.
\end{equation}

There's a problem though with the $\mathbf{E}$ solution because the vector component is not orthogonal to the gradient of $S$. The simplest fix for this would be to prohibit charges, i.e. $\rho=0$. Maybe we can allow dipole charges.



\section{Second Order System}

An alternative to the above solution procedure is to solve the system as far as possible before splitting it by powers of $\omega$. Solving \eqref{eq:max3} for $\mathbf{h}$ and plugging the result into \eqref{eq:max4} gives



\section{Starting with Wave Equation}


\end{document}
