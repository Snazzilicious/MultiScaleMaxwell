
\documentclass{article}

\usepackage[a4paper, total={6in, 8in}]{geometry}
\usepackage{amsmath}
\usepackage{amsfonts}
\usepackage{amsthm}
\usepackage{indentfirst}
\usepackage{hyperref}

\newcommand{\norm}[1]{||#1||}

\theoremstyle{plain}
\newtheorem*{remark}{Remark}

\begin{document}

\section{Intro}\label{sec:intro}

We begin with Maxwell's equations in matter
\begin{equation}
	\nabla\cdot\mathbf{D} = \rho
\end{equation}
\begin{equation}
	\nabla\cdot\mathbf{B} = 0
\end{equation}
\begin{equation}
	\nabla\times\mathbf{E} = -\frac{d\mathbf{B}}{dt}
\end{equation}
\begin{equation}
	\nabla\times\mathbf{H} = \mathbf{J} + \frac{d\mathbf{D}}{dt}.
\end{equation}
We assume linearly polarizable media,
\begin{equation}
	\mathbf{D} = \epsilon\mathbf{E}
\end{equation}
\begin{equation}
	\mathbf{H} = \frac{1}{\mu}\mathbf{B},
\end{equation}
where $\epsilon=\epsilon(x)$ and $\mu=\mu(x)$. We also assume the solution to be time-harmonic (e.g. $\mathbf{E}(t,x)=\mathbf{E}(x)e^{-j\omega t}$) and so replace time derivative operators with $-j\omega$. We furthermore assume $\mathbf{E}(x) = \mathbf{e}(x)e^{j\omega S(x)}$ and $\mathbf{H}(x) = \mathbf{h}(x)e^{j\omega S(x)}$. The resulting equations are
%\begin{equation}\label{eq:max1}
%	\nabla\cdot\left(\epsilon\mathbf{E}\right) = \rho
%\end{equation}
%\begin{equation}\label{eq:max2}
%	\nabla\cdot\mathbf{B} = 0
%\end{equation}
%\begin{equation}\label{eq:max3}
%	\nabla\times\mathbf{E} = -j\omega\mathbf{B}
%\end{equation}
%\begin{equation}\label{eq:max4}
%	\nabla\times\left(\frac{1}{\mu}\mathbf{B}\right) = \mathbf{J} + j\omega\epsilon\mathbf{E}.
%\end{equation}
% NOTE: Using SI units instead of Gauss
\begin{equation}\label{eq:max1}
	\mathbf{e}\cdot\nabla S = \frac{1}{j\omega\epsilon}\left( \rho e^{-j\omega S(x)} - \nabla\cdot\epsilon\mathbf{e} \right)
\end{equation}
\begin{equation}\label{eq:max2}
	\mathbf{h}\cdot\nabla S = -\frac{1}{j\omega\mu}\nabla\cdot\mu\mathbf{h}
\end{equation}
\begin{equation}\label{eq:max3}
	\nabla S\times\mathbf{e} - \mu\mathbf{h} = -\frac{1}{j\omega}\nabla\times\mathbf{e}
\end{equation}
\begin{equation}\label{eq:max4}
	\nabla S\times\mathbf{h} + \epsilon\mathbf{e} = \frac{1}{j\omega}\left( \mathbf{J}e^{-j\omega S(x)} - \nabla\times\mathbf{h} \right).
\end{equation}
So far this closely follows Wolf and Born, except using SI units instead of Gauss, and retaining charges and currents.


\section{First Order System}

Leaving the system of equations as is, we have two systems
\begin{equation}
	\omega^0:
	\begin{cases}
		\mathbf{e}\cdot\nabla S = 0 \\
		\mathbf{h}\cdot\nabla S = 0 \\
		\nabla S\times\mathbf{e} - \mu\mathbf{h} = 0 \\
		\nabla S\times\mathbf{h} + \epsilon\mathbf{e} = 0
	\end{cases}
\end{equation}
\begin{equation}
	\omega^1:
	\begin{cases}
		\nabla\cdot\epsilon\mathbf{e} = \rho' \\
		\nabla\cdot\mu\mathbf{h} = 0 \\
		\nabla\times\mathbf{e} = 0 \\
		\nabla\times\mathbf{h} = \mathbf{J}',
	\end{cases}
\end{equation}
where we have defined $\rho' = \rho e^{-j\omega S(x)}$ and $\mathbf{J}' = \mathbf{J}e^{-j\omega S(x)}$.

To solve the first system, we first solve the third equation for $\mathbf{h}$, then plug that into the fourth equation. This gives
\begin{equation}
	\frac{1}{\mu}\left[ \left(\mathbf{e}\cdot\nabla S\right)\nabla S - \norm{\nabla S}^2\mathbf{e}\right] + \epsilon\mathbf{e} = 0.
\end{equation}
Invoking the first equation results in
\begin{equation}
	\left(\mu\epsilon - \norm{\nabla S}^2\right)\mathbf{e} = 0,
\end{equation}
or
\begin{equation}
	\norm{\nabla S}^2 = \mu\epsilon.
\end{equation}
This is the eikonal equation.

Inspecting the second system, we see that it is the plain electro/magneto-statics equations, that is to say that the wave part of the solution has been removed. Hopefully this is easier to solve than the EFIE/MFIE problem.



\section{Second Order System}

An alternative to the above solution procedure is to solve the system as far as possible before splitting it by powers of $\omega$. Solving \eqref{eq:max3} for $\mathbf{h}$ and plugging the result into \eqref{eq:max4} gives




\end{document}
