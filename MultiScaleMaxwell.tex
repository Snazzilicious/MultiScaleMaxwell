
\documentclass{article}

\usepackage[a4paper, total={6in, 8in}]{geometry}
\usepackage{amsmath}
\usepackage{amsfonts}
\usepackage{amsthm}
\usepackage{indentfirst}
\usepackage{hyperref}

\newcommand{\norm}[1]{||#1||}

\theoremstyle{plain}
\newtheorem*{remark}{Remark}

\begin{document}

\section{Intro}\label{sec:intro}

We begin with Maxwell's equations in matter
\begin{equation}
	\nabla\cdot\mathbf{D} = \rho
\end{equation}
\begin{equation}
	\nabla\cdot\mathbf{B} = 0
\end{equation}
\begin{equation}
	\nabla\times\mathbf{E} = -\frac{d\mathbf{B}}{dt}
\end{equation}
\begin{equation}
	\nabla\times\mathbf{H} = \mathbf{J} + \frac{d\mathbf{D}}{dt}.
\end{equation}
We assume linearly polarizable media,
\begin{equation}
	\mathbf{D} = \epsilon\mathbf{E}
\end{equation}
\begin{equation}
	\mathbf{H} = \frac{1}{\mu}\mathbf{B},
\end{equation}
where $\epsilon=\epsilon(x)$ and $\mu=\mu(x)$. We also assume the solution to be time-harmonic and so replace time derivative operators with $-j\omega$. We furthermore assume $\mathbf{E}(x) = \mathbf{e}(x)e^{j\omega S(x)}$ and $\mathbf{H}(x) = \mathbf{h}(x)e^{j\omega S(x)}$. The resulting equations are
%\begin{equation}\label{eq:max1}
%	\nabla\cdot\left(\epsilon\mathbf{E}\right) = \rho
%\end{equation}
%\begin{equation}\label{eq:max2}
%	\nabla\cdot\mathbf{B} = 0
%\end{equation}
%\begin{equation}\label{eq:max3}
%	\nabla\times\mathbf{E} = -j\omega\mathbf{B}
%\end{equation}
%\begin{equation}\label{eq:max4}
%	\nabla\times\left(\frac{1}{\mu}\mathbf{B}\right) = \mathbf{J} + j\omega\epsilon\mathbf{E}.
%\end{equation}
% NOTE: Using SI units instead of Gauss
\begin{equation}\label{eq:max1}
	\mathbf{e}\cdot\nabla S = -\frac{1}{j\omega}\left( \mathbf{e}\cdot\nabla\log{\epsilon} + \nabla\cdot\mathbf{e}-\rho \right)
\end{equation}
\begin{equation}\label{eq:max2}
	\mathbf{h}\cdot\nabla S = -\frac{1}{j\omega}\left( \mathbf{h}\cdot\nabla\log{\mu} + \nabla\cdot\mathbf{h}\right)
\end{equation}
\begin{equation}\label{eq:max3}
	\nabla S\times\mathbf{e} - \mu\mathbf{h} = -\frac{1}{j\omega}\nabla\times\mathbf{e}
\end{equation}
\begin{equation}\label{eq:max4}
	\nabla S\times\mathbf{h} + \epsilon\mathbf{e} = -\frac{1}{j\omega}\left( \nabla\times\mathbf{h} - \mathbf{J} \right).
\end{equation}
So far this closely follows Wolf and Born, except using SI units instead of Gauss, and retaining charges and currents.

\section{Scale Splitting}\label{sec:scaleSplit}

We assume that $\omega$ is much larger than the characteristic length of the domain, so that when the problem is non-dimensionalized, $\omega >> 1$. Stemming from this assumption, we replace $\omega$ with $1/\delta$ for $\delta<<1$.
\begin{equation}\label{eq:max1a}
	\nabla\cdot\left(\epsilon\mathbf{E}\right) = \rho
\end{equation}
\begin{equation}\label{eq:max2a}
	\nabla\cdot\mathbf{B} = 0
\end{equation}
\begin{equation}\label{eq:max3a}
	\delta\nabla\times\mathbf{E} = -j\mathbf{B}
\end{equation}
\begin{equation}\label{eq:max4a}
	\delta\nabla\times\left(\frac{1}{\mu}\mathbf{B}\right) = \delta\mathbf{J} + j\epsilon\mathbf{E}.
\end{equation}

Although we assume the domain to be large relative to $\omega$, we assume there to be object(s) in the domain closer to the scale of $\omega$. To resolve these affects, we separate the problem into two scales. The standard scale is that for which $\omega$ is large, and one for which it is smaller. This separation can be represented by introducing a new function.

Suppose $\exists \mathbf{F}(y,z)$ and $\exists \mathbf{C}(y,z)$ s.t. $\mathbf{E}(x) = \mathbf{F}(x,\frac{x}{\delta})$ and $\mathbf{B}(x) = \mathbf{C}(x,\frac{x}{\delta})$. We would then have $\nabla_x = \nabla_y + \frac{1}{\delta}\nabla_z$. Plugging this into the system gives us
\begin{equation}\label{eq:max1b}
	\delta\nabla_y\cdot\left(\epsilon\mathbf{F}\right) + \nabla_z\cdot\left(\epsilon\mathbf{F}\right) = \delta\rho
\end{equation}
\begin{equation}\label{eq:max2b}
	\delta\nabla_y\cdot\mathbf{C} + \nabla_z\cdot\mathbf{C} = 0
\end{equation}
\begin{equation}\label{eq:max3b}
	\delta\nabla_y\times\mathbf{F} + \nabla_z\times\mathbf{F} = -j\mathbf{C}
\end{equation}
\begin{equation}\label{eq:max4b}
	\delta\nabla_y\times\left(\frac{1}{\mu}\mathbf{C}\right) + \nabla_z\times\left(\frac{1}{\mu}\mathbf{C}\right) = \delta\mathbf{J} + j\epsilon\mathbf{F}.
\end{equation}
Assuming that $\epsilon$ and $\mu$ are functions of $y$ only, we have
\begin{equation}\label{eq:max1c}
	\delta\nabla_y\cdot\left(\epsilon\mathbf{F}\right) + \epsilon\nabla_z\cdot\mathbf{F} = \delta\rho
\end{equation}
\begin{equation}\label{eq:max2c}
	\delta\nabla_y\cdot\mathbf{C} + \nabla_z\cdot\mathbf{C} = 0
\end{equation}
\begin{equation}\label{eq:max3c}
	\delta\nabla_y\times\mathbf{F} + \nabla_z\times\mathbf{F} = -j\mathbf{C}
\end{equation}
\begin{equation}\label{eq:max4c}
	\delta\nabla_y\times\left(\frac{1}{\mu}\mathbf{C}\right) + \frac{1}{\mu}\nabla_z\times\mathbf{C} = \delta\mathbf{J} + j\epsilon\mathbf{F}.
\end{equation}




The solution is assumed to be of the form $\mathbf{F}_0 + \delta \mathbf{F}_1 + \delta^2\mathbf{F}_2 + ...$, which can be plugged into the equation to produce the system
\begin{equation}
	\delta^0:
	\begin{cases}
		\nabla_z\cdot\mathbf{F}_0 = 0 \\
		\nabla_z\cdot\mathbf{C}_0 = 0 \\
		\nabla_z\times\mathbf{F}_0 = -j\mathbf{C}_0 \\
		\nabla_z\times\mathbf{C}_0 = j\mu\epsilon\mathbf{F}_0
	\end{cases}
\end{equation}
\begin{equation}
	\delta^1:
	\begin{cases}
		\nabla_z\cdot\mathbf{F}_1 = \frac{\rho}{\epsilon} - \frac{1}{\epsilon}\nabla_y\cdot\left(\epsilon\mathbf{F}_0\right)\\
		\nabla_z\cdot\mathbf{C}_1 = -\nabla_y\cdot\mathbf{C}_0 \\
		\nabla_z\times\mathbf{F}_1 = -j\mathbf{C}_1 - \nabla_y\times\mathbf{F}_0\\
		\nabla_z\times\mathbf{C}_1 = \mu\mathbf{J} + j\mu\epsilon\mathbf{F}_1 - \mu\nabla_y\times\left(\frac{1}{\mu}\mathbf{C}_0\right)
	\end{cases}
\end{equation}
\begin{equation}
	\delta^2:
	\begin{cases}
		\nabla_z\cdot\mathbf{F}_2 = -\frac{1}{\epsilon}\nabla_y\cdot\left(\epsilon\mathbf{F}_1\right)\\
		\nabla_z\cdot\mathbf{C}_2 = -\nabla_y\cdot\mathbf{C}_1 \\
		\nabla_z\times\mathbf{F}_2 = -j\mathbf{C}_2 - \nabla_y\times\mathbf{F}_1\\
		\nabla_z\times\mathbf{C}_2 = j\mu\epsilon\mathbf{F}_2 - \mu\nabla_y\times\left(\frac{1}{\mu}\mathbf{C}_1\right)
	\end{cases}
\end{equation}



\section{Solution}\label{sec:sol}

The gist is that the first equation yields a small scale plane wave with a long scale amplitude, $\mathbf{f}_0(y)e^{j\sqrt{\mu\epsilon}\left(\mathbf{v}\cdot z\right)}$. The amplitude, $\mathbf{f}_0(y)$, is solved for by setting
\begin{equation*}
	2\nabla_y\cdot\nabla_z\mathbf{F}_0 + \nabla_z\left( \mathbf{F}_0\cdot\nabla_y\log{\epsilon} \right) + \nabla_y\log{\mu}\times\nabla_z\times\mathbf{F}_0 = 0
\end{equation*}
from the second equation. This is term-by-term and step-by-step analogous to Wolf and Born's derivation of physical optics in Principles of Optics section 3.1.3.

The leftovers from the second equation are the EFIE problem which has solution along the lines of
\begin{equation*}
	\int\mathbf{f}_1(y,z_0) \frac{ e^{j\sqrt{\mu\epsilon}\norm{z-z_0}} }{ \norm{z-z_0} } dz_0,
\end{equation*}
Where $\mathbf{f}_1(y*,z_0) = \left( j\mu\mathbf{J} + \frac{1}{\epsilon}\nabla_z\rho \right)$ probably. We then solve a similar ODE for $\mathbf{f}_1(y,z_0)$ as we did for $\mathbf{f}_0(y)$ and neglect any higher order terms.

The final soution is then something like
%\begin{equation*}
%	\mathbf{f}_0(x)e^{j\sqrt{\mu\epsilon}\omega\left(\mathbf{v}\cdot{x}\right)} + \frac{1}{\omega}\int\mathbf{f}_1(x,x_0) \frac{ e^{j\sqrt{\mu\epsilon}\omega\norm{x-x_0}} }{ \omega\norm{x-x_0} } \omega dx_0.
%\end{equation*}
\begin{equation*}
	\mathbf{f}_0(x)e^{j\sqrt{\mu\epsilon}\omega\left(\mathbf{v}\cdot{x}\right)} + \int\mathbf{f}_1(x,x_0) \frac{ e^{j\sqrt{\mu\epsilon}\omega\norm{x-x_0}} }{ \omega\norm{x-x_0} } dx_0.
\end{equation*}



\subsection{Order 0}

The first system yields a wave equation by taking the curl w.r.t. $z$ of the third equation and plugging the fourth equation into it. This equation is
\begin{equation*}
	\nabla_z\times\nabla_z\times\mathbf{F}_0 = \mu\epsilon\mathbf{F}_0,
\end{equation*}
\begin{equation}
	-\nabla_z^2\mathbf{F}_0 + \nabla_z\left(\nabla_z\cdot\mathbf{F}_0\right) = \mu\epsilon\mathbf{F}_0.
\end{equation}
Invoking the first equation in the system, this is now
\begin{equation}
	\nabla_z^2\mathbf{F}_0 + \mu\epsilon\mathbf{F}_0 = 0.
\end{equation}
This is solved by
\begin{equation}
	\mathbf{F}_0(y,z) = \mathbf{f}_0(y)e^{j\left(\mathbf{v}(y)\cdot z\right)},
\end{equation}
where $\norm{\mathbf{v}(y)}^2=\mu\epsilon$. We note this satisfies the eikonal equation,
\begin{equation}
	\norm{\nabla_z \left(\mathbf{v}\cdot z\right) }^2 = \mu\epsilon = n^2.
\end{equation}

To ensure that the first equation is satisfied, we plug the solution in and get
\begin{equation}
	e^{j\left(\mathbf{v}(y)\cdot z\right)}\mathbf{f}_0(y)\cdot\mathbf{v}(y) = 0
\end{equation}
which implies
\begin{equation}
	\mathbf{f}_0(y)\cdot\mathbf{v}(y) = 0.
\end{equation}

The third equation gives us
\begin{equation}
	\mathbf{C}_0 = -e^{j\left(\mathbf{v}(y)\cdot z\right)}\mathbf{v}(y)\times\mathbf{f}_0(y),
\end{equation}
which automatically satisfies the second equation.



\subsection{Order 1}

The second system can also produce a wave equation by taking the curl w.r.t. $z$ of the third equation and plugging the fourth equation into it. This equation is
\begin{equation}
	-\nabla_z^2\mathbf{F}_1 + \nabla_z\left(\nabla_z\cdot\mathbf{F}_1\right) = -j\mu\mathbf{J} + \mu\epsilon\mathbf{F}_1 + j\mu\nabla_y\times\left(\frac{1}{\mu}\mathbf{C}_0\right) - \nabla_z\times\nabla_y\times\mathbf{F}_0.
\end{equation}
Plugging in the first equation in the system gives
\begin{equation}
	\nabla_z^2\mathbf{F}_1 + \mu\epsilon\mathbf{F}_1 = j\mu\mathbf{J} + \frac{1}{\epsilon}\nabla_z\rho - \frac{1}{\epsilon}\nabla_z\left(\nabla_y\cdot\left(\epsilon\mathbf{F}_0\right)\right) - j\mu\nabla_y\times\left(\frac{1}{\mu}\mathbf{C}_0\right) + \nabla_z\times\nabla_y\times\mathbf{F}_0.
\end{equation}
We plug the Order 0 solution into the last 3 terms which yields the following.

\subsection{Term 1}

\begin{equation}
	-\frac{1}{\epsilon}\nabla_z\left(\nabla_y\cdot\left(\epsilon\mathbf{F}_0\right)\right) = -\frac{1}{\epsilon}\nabla_z\left(\nabla_y\cdot\left( \epsilon\mathbf{f}_0(y)e^{j\left(\mathbf{v}(y)\cdot z\right)} \right)\right)
\end{equation}
\begin{equation}
	= -\frac{1}{\epsilon}\nabla_z\left( e^{j\left(\mathbf{v}(y)\cdot z\right)}\mathbf{f}_0(y)\cdot\nabla_y\epsilon + \epsilon je^{j\left(\mathbf{v}(y)\cdot z\right)}\mathbf{f}_0(y)\cdot\nabla_y(\mathbf{v}(y)\cdot z) + \epsilon e^{j\left(\mathbf{v}(y)\cdot z\right)}\nabla_y\cdot\mathbf{f}_0(y) \right)
\end{equation}
\begin{equation}
	= -\nabla_z e^{j\left(\mathbf{v}(y)\cdot z\right)}\left( \mathbf{f}_0(y)\cdot\nabla_y\log{\epsilon} + j\mathbf{f}_0(y)\cdot\nabla_y(\mathbf{v}(y)\cdot z) + \nabla_y\cdot\mathbf{f}_0(y) \right)
\end{equation}
\begin{equation}
	= -\left[ je^{j\left(\mathbf{v}(y)\cdot z\right)}\left( \mathbf{f}_0(y)\cdot\nabla_y\log{\epsilon} + j\mathbf{f}_0(y)\cdot\nabla_y(\mathbf{v}(y)\cdot z) + \nabla_y\cdot\mathbf{f}_0(y) \right)\mathbf{v}(y) + je^{j\left(\mathbf{v}(y)\cdot z\right)}\mathbf{f}_0(y)\cdot\nabla_y\mathbf{v}(y) \right]
\end{equation}
\begin{equation}
	= -je^{j\left(\mathbf{v}(y)\cdot z\right)}\left[ \left( \mathbf{f}_0(y)\cdot\nabla_y\log{\epsilon} + j\mathbf{f}_0(y)\cdot\nabla_y(\mathbf{v}(y)\cdot z) + \nabla_y\cdot\mathbf{f}_0(y) \right)\mathbf{v}(y) + \mathbf{f}_0(y)\cdot\nabla_y\mathbf{v}(y) \right]
\end{equation}


\subsection{Term 2}

\begin{equation}
	-j\mu\nabla_y\times\left(\frac{1}{\mu}\mathbf{C}_0\right) = j\mu\nabla_y\times\left( \frac{1}{\mu}e^{j\left(\mathbf{v}(y)\cdot z\right)}\mathbf{v}(y)\times\mathbf{f}_0(y) \right)
\end{equation}
\begin{equation}
	= j\mu\left[ \frac{1}{\mu}e^{j\left(\mathbf{v}(y)\cdot z\right)}\nabla_y\times(\mathbf{v}(y)\times\mathbf{f}_0(y)) + \left( e^{j\left(\mathbf{v}(y)\cdot z\right)}\nabla_y\frac{1}{\mu} + \frac{1}{\mu}je^{j\left(\mathbf{v}(y)\cdot z\right)}\nabla_y(\mathbf{v}(y)\cdot z) \right)\times(\mathbf{v}(y)\times\mathbf{f}_0(y)) \right]
\end{equation}
\begin{equation}
	= je^{j\left(\mathbf{v}(y)\cdot z\right)}\left[ \nabla_y\times(\mathbf{v}(y)\times\mathbf{f}_0(y)) + \left( -\nabla_y\log{\mu} + j\nabla_y(\mathbf{v}(y)\cdot z) \right)\times(\mathbf{v}(y)\times\mathbf{f}_0(y)) \right]
\end{equation}
\begin{multline}
	= je^{j\left(\mathbf{v}(y)\cdot z\right)} \Big[ \mathbf{v}(y)\nabla_y\cdot\mathbf{f}_0(y) - \mathbf{f}_0(y)\nabla_y\cdot\mathbf{v}(y) + \mathbf{f}_0(y)\cdot\nabla_y\mathbf{v}(y) - \mathbf{v}(y)\cdot\nabla_y\mathbf{f}_0(y) \\ + \left( -\mathbf{f}_0(y)\cdot\nabla_y\log{\mu} + j\mathbf{f}_0(y)\cdot\nabla_y(\mathbf{v}(y)\cdot z) \right)\mathbf{v}(y) - \left( -\mathbf{v}(y)\cdot\nabla_y\log{\mu} + j\mathbf{v}(y)\cdot\nabla_y(\mathbf{v}(y)\cdot z) \right)\mathbf{f}_0(y) \Big]
\end{multline}


\subsection{Term 3}

\begin{equation}
	\nabla_z\times\nabla_y\times\mathbf{F}_0 = \nabla_z\times\nabla_y\times\mathbf{f}_0(y)e^{j\left(\mathbf{v}(y)\cdot z\right)}
\end{equation}
\begin{equation}
	= \nabla_z\times\left[ \nabla_y e^{j\left(\mathbf{v}(y)\cdot z\right)}\times\mathbf{f}_0(y) + e^{j\left(\mathbf{v}(y)\cdot z\right)}\nabla_y\times\mathbf{f}_0(y) \right]
\end{equation}
\begin{equation}
	= \nabla_z\times\left[ je^{j\left(\mathbf{v}(y)\cdot z\right)}\nabla_y(\mathbf{v}(y)\cdot z)\times\mathbf{f}_0(y) + e^{j\left(\mathbf{v}(y)\cdot z\right)}\nabla_y\times\mathbf{f}_0(y) \right]
\end{equation}
\begin{equation}
	= j^2e^{j\left(\mathbf{v}(y)\cdot z\right)}\mathbf{v}(y)\times\nabla_y(\mathbf{v}(y)\cdot z)\times\mathbf{f}_0(y) + je^{j\left(\mathbf{v}(y)\cdot z\right)}\nabla_z\times\nabla_y(\mathbf{v}(y)\cdot z)\times\mathbf{f}_0(y) + je^{j\left(\mathbf{v}(y)\cdot z\right)}\mathbf{v}(y)\times\nabla_y\times\mathbf{f}_0(y)
\end{equation}
\begin{multline}
	= j^2e^{j\left(\mathbf{v}(y)\cdot z\right)}\left[ (\mathbf{v}(y)\cdot\mathbf{f}_0(y))\nabla_y(\mathbf{v}(y)\cdot z) - (\mathbf{v}(y)\cdot\nabla_y(\mathbf{v}(y)\cdot z))\mathbf{f}_0(y) \right] \\ + je^{j\left(\mathbf{v}(y)\cdot z\right)}\left[ -\mathbf{f}_0(y)\nabla_z\cdot\nabla_y(\mathbf{v}(y)\cdot z) + \mathbf{f}_0(y)\cdot\nabla_z\nabla_y(\mathbf{v}(y)\cdot z) \right] \\ + je^{j\left(\mathbf{v}(y)\cdot z\right)}\mathbf{v}(y)\times\nabla_y\times\mathbf{f}_0(y)
\end{multline}
\begin{multline}
	= je^{j\left(\mathbf{v}(y)\cdot z\right)} \Big[ -j(\mathbf{v}(y)\cdot\nabla_y(\mathbf{v}(y)\cdot z))\mathbf{f}_0(y) \\ -\mathbf{f}_0(y)\nabla_z\cdot\nabla_y(\mathbf{v}(y)\cdot z) + \mathbf{f}_0(y)\cdot\nabla_z\nabla_y(\mathbf{v}(y)\cdot z) \\ + \mathbf{v}(y)\times\nabla_y\times\mathbf{f}_0(y) \Big]
\end{multline}
\begin{multline}
	= je^{j\left(\mathbf{v}(y)\cdot z\right)} \Big[ -j(\mathbf{v}(y)\cdot\nabla_y(\mathbf{v}(y)\cdot z))\mathbf{f}_0(y) -\mathbf{f}_0(y)\nabla_y\cdot\mathbf{v}(y) + \mathbf{f}_{0,l}(y)\frac{\partial\mathbf{v}_l(y)}{\partial y_k} \\ - \mathbf{f}_0(y)\cdot\nabla_y\mathbf{v}(y) - \mathbf{v}(y)\cdot\nabla_y\mathbf{f}_0(y) - \mathbf{f}_0(y)\times\nabla_y\times\mathbf{v}(y) \Big]
\end{multline}


\subsection{Putting Together}

The combined terms are
\begin{multline}
	je^{j\left(\mathbf{v}(y)\cdot z\right)}\Big[ \\
		-\left( \mathbf{f}_0(y)\cdot\nabla_y\log{\epsilon} + j\mathbf{f}_0(y)\cdot\nabla_y(\mathbf{v}(y)\cdot z) + \nabla_y\cdot\mathbf{f}_0(y) \right)\mathbf{v}(y) - \mathbf{f}_0(y)\cdot\nabla_y\mathbf{v}(y) \\
		+ \mathbf{v}(y)\nabla_y\cdot\mathbf{f}_0(y) - \mathbf{f}_0(y)\nabla_y\cdot\mathbf{v}(y) + \mathbf{f}_0(y)\cdot\nabla_y\mathbf{v}(y) - \mathbf{v}(y)\cdot\nabla_y\mathbf{f}_0(y) \\ + \left( -\mathbf{f}_0(y)\cdot\nabla_y\log{\mu} + j\mathbf{f}_0(y)\cdot\nabla_y(\mathbf{v}(y)\cdot z) \right)\mathbf{v}(y) - \left( -\mathbf{v}(y)\cdot\nabla_y\log{\mu} + j\mathbf{v}(y)\cdot\nabla_y(\mathbf{v}(y)\cdot z) \right)\mathbf{f}_0(y) \\
		-j(\mathbf{v}(y)\cdot\nabla_y(\mathbf{v}(y)\cdot z))\mathbf{f}_0(y) -\mathbf{f}_0(y)\nabla_y\cdot\mathbf{v}(y) + \mathbf{f}_{0,l}(y)\frac{\partial\mathbf{v}_l(y)}{\partial y_k} \\ - \mathbf{f}_0(y)\cdot\nabla_y\mathbf{v}(y) - \mathbf{v}(y)\cdot\nabla_y\mathbf{f}_0(y) - \mathbf{f}_0(y)\times\nabla_y\times\mathbf{v}(y)
	\\ \Big]
\end{multline}


\subsection{Secular terms}

To prevent secular terms, we require that
\begin{equation}
	\mathbf{v}\cdot\nabla_y\mathbf{f}_0(y)  + \left[ \left(\frac{1}{\sqrt{\mu\epsilon}}+j\left(\mathbf{v}\cdot z\right)\right)\left(\mathbf{v}\cdot\nabla_y\sqrt{\mu\epsilon}\right) - \left(\mathbf{v}\cdot\nabla_y\log{\sqrt{\mu}}\right)\right]\mathbf{f}_0(y)  + \left( \mathbf{f}_0(y)\cdot\nabla_y\log{\sqrt{\mu\epsilon}} \right)\mathbf{v} = 0.
\end{equation}
Introducing $\mathbf{g}_0(r)$ with $\frac{dy_i}{dr}=v_i$, such that $\mathbf{g}_0(v\cdot y) = \mathbf{f}_0(y)$, this equation is
\begin{equation}
	\frac{d}{dr}\mathbf{g}_0(r)  + \left[ \left(\frac{1}{\sqrt{\mu\epsilon}}+j\left(\mathbf{v}\cdot z\right)\right)\left(\frac{d}{dr}\sqrt{\mu\epsilon}\right) - \left(\frac{d}{dr}\log{\sqrt{\mu}}\right) \right]\mathbf{g}_0(r)  + \left( \mathbf{g}_0(r)\cdot\nabla_y\log{\sqrt{\mu\epsilon}} \right)\mathbf{v} = 0.
\end{equation}
It is convenient to solve for the magnitude and direction of $\mathbf{g}_0(r)$ separately. To solve for the magnitude, we scalar multiply with $\mathbf{g}^*_0(r)$, then add the complex conjugate of the result to itself. This produces
\begin{equation}
	\frac{d}{dr}\mathbf{g}_0(r)\cdot\mathbf{g}^*_0(r) + 2\left[ \frac{1}{\sqrt{\mu\epsilon}}\frac{d}{dr}\sqrt{\mu\epsilon} - \left(\frac{d}{dr}\log{\sqrt{\mu}}\right) \right]\mathbf{g}_0(r)\cdot\mathbf{g}^*_0(r) = 0.
\end{equation}
\begin{equation}
	\frac{d}{dr}\mathbf{g}_0(r)\cdot\mathbf{g}^*_0(r) + \frac{1}{\epsilon}\frac{d\epsilon}{dr}\mathbf{g}_0(r)\cdot\mathbf{g}^*_0(r) = 0.
\end{equation}
\begin{equation}
	\norm{\mathbf{g}_0(r)}^2 = \frac{K_0}{ \epsilon(r) }.
\end{equation}

To solve for the direction, we plug in $\mathbf{g}_0(r) = \sqrt{\mathbf{g}_0(r)\cdot\mathbf{g}^*_0(r)}\mathbf{u}_0(r)$, where $\mathbf{u}_0(r)$ is a unit vector.
\begin{multline}
	\mathbf{u}_0(r)\frac{d}{dr}\sqrt{\mathbf{g}_0(r)\cdot\mathbf{g}^*_0(r)} + \sqrt{\mathbf{g}_0(r)\cdot\mathbf{g}^*_0(r)}\frac{d}{dr}\mathbf{u}_0(r) \\ + \left[ \left(\frac{1}{\sqrt{\mu\epsilon}}+j\left(\mathbf{v}\cdot z\right)\right)\left(\frac{d}{dr}\sqrt{\mu\epsilon}\right) - \left(\frac{d}{dr}\log{\sqrt{\mu}}\right) \right]\sqrt{\mathbf{g}_0(r)\cdot\mathbf{g}^*_0(r)}\mathbf{u}_0(r) \\ + \sqrt{\mathbf{g}_0(r)\cdot\mathbf{g}^*_0(r)}\left( \mathbf{u}_0(r)\cdot\nabla_y\log{\sqrt{\mu\epsilon}} \right)\mathbf{v} = 0.
\end{multline}
\begin{multline}
	\mathbf{u}_0(r)\frac{1}{2\sqrt{\mathbf{g}_0(r)\cdot\mathbf{g}^*_0(r)}}\frac{d}{dr}\mathbf{g}_0(r)\cdot\mathbf{g}^*_0(r) + \sqrt{\mathbf{g}_0(r)\cdot\mathbf{g}^*_0(r)}\frac{d}{dr}\mathbf{u}_0(r) \\ + \left[ \frac{d}{dr}\log{\sqrt{\epsilon}} + j\left(\mathbf{v}\cdot z\right)\frac{d}{dr}\sqrt{\mu\epsilon} \right]\sqrt{\mathbf{g}_0(r)\cdot\mathbf{g}^*_0(r)}\mathbf{u}_0(r) \\ + \sqrt{\mathbf{g}_0(r)\cdot\mathbf{g}^*_0(r)}\left( \mathbf{u}_0(r)\cdot\nabla_y\log{\sqrt{\mu\epsilon}} \right)\mathbf{v} = 0.
\end{multline}
\begin{multline}
	\mathbf{u}_0(r)\frac{d}{dr}\mathbf{g}_0(r)\cdot\mathbf{g}^*_0(r) + 2\mathbf{g}_0(r)\cdot\mathbf{g}^*_0(r)\frac{d}{dr}\mathbf{u}_0(r) \\ + 2\left[ \frac{d}{dr}\log{\sqrt{\epsilon}} + j\left(\mathbf{v}\cdot z\right)\frac{d}{dr}\sqrt{\mu\epsilon} \right]\mathbf{g}_0(r)\cdot\mathbf{g}^*_0(r)\mathbf{u}_0(r) \\ + 2\mathbf{g}_0(r)\cdot\mathbf{g}^*_0(r)\left( \mathbf{u}_0(r)\cdot\nabla_y\log{\sqrt{\mu\epsilon}} \right)\mathbf{v} = 0.
\end{multline}
\begin{multline}
	\mathbf{u}_0(r)\left[ \frac{d}{dr}\mathbf{g}_0(r)\cdot\mathbf{g}^*_0(r) + \frac{1}{\epsilon}\frac{d\epsilon}{dr}\mathbf{g}_0(r)\cdot\mathbf{g}^*_0(r) \right] \\ + 2\mathbf{g}_0(r)\cdot\mathbf{g}^*_0(r)\frac{d}{dr}\mathbf{u}_0(r) + 2\left[ j\left(\mathbf{v}\cdot z\right)\frac{d}{dr}\sqrt{\mu\epsilon} \right]\mathbf{g}_0(r)\cdot\mathbf{g}^*_0(r)\mathbf{u}_0(r) \\ + 2\mathbf{g}_0(r)\cdot\mathbf{g}^*_0(r)\left( \mathbf{u}_0(r)\cdot\nabla_y\log{\sqrt{\mu\epsilon}} \right)\mathbf{v} = 0.
\end{multline}
The first term is zero from before,
\begin{equation}
	\frac{d}{dr}\mathbf{u}_0(r) \\ + \left[ j\left(\mathbf{v}\cdot z\right)\frac{d}{dr}\sqrt{\mu\epsilon} \right]\mathbf{u}_0(r) \\ + \left( \mathbf{u}_0(r)\cdot\nabla_y\log{\sqrt{\mu\epsilon}} \right)\mathbf{v} = 0.
\end{equation}



The second equation is now
\begin{equation}
	\mu\epsilon\mathbf{F}_1 + \nabla_z^2\mathbf{F}_1 = \left( j\mu\mathbf{J} + \frac{1}{\epsilon}\nabla_z\rho \right),
\end{equation}
which is the EFIE problem.

To be continued ...


\end{document}
